\documentclass[12pt,a4paper]{article}
\usepackage{natbib}         % Pour la bibliographie
\usepackage{url}            % Pour citer les adresses web
\usepackage[T1]{fontenc}    % Encodage des accents
\usepackage[utf8]{inputenc} % Lui aussi
\usepackage[frenchb]{babel} % Pour la traduction française
\usepackage{numprint}       % Histoire que les chiffres soient bien

\usepackage{amsmath}        % La base pour les maths
\usepackage{mathrsfs}       % Quelques symboles supplémentaires
\usepackage{amssymb}        % encore des symboles.
\usepackage{amsfonts}       % Des fontes, eg pour \mathbb.

\usepackage[svgnames]{xcolor} % De la couleur
\usepackage{geometry}       % Gérer correctement la taille

\usepackage{tikz, pgfplots}
\usetikzlibrary{positioning, shapes, arrows, shadows}

\title{Test Tikz}
\author{Me}

\begin{document}

\section{Structure globale de l'architecture transformer}

Le schéma de l'achitecture dans le cas de la génération en mode inférence est le suivant :

\begin{center}
\begin{tikzpicture}[
Block/.style =  {rectangle,  rounded corners, draw=black!100, fill =white!0, thick, minimum width = 80px},
Eblock/.style = {rectangle, rounded corners, draw=black!40!green, fill =green!10, thick, minimum height = 50px, minimum width = 100px},
Dblock/.style = {rectangle, rounded corners, draw=black!40!blue, fill =blue!10, thick, minimum height = 50px, minimum width = 100px},
Vect/.style = {circle, draw=purple!80!red, thick, font=\footnotesize, text width=8px, align=left},
Fleche1/.style = {->, thick, shorten <=5px, shorten >=5px},
Fleche2/.style= {->,orange, thick, shorten <=5px, shorten >= 5px}
]

%Nodes

%Encoder Blocks
\node[Block] (Tok) {Tokenizer};
\node[Block, align=center] (EPE)  [above=15px of Tok] {Embedding + \\Positional Encoding};
\node[Eblock] (Enc1) [above=15px of EPE] {Encoder $1$};
\node[Eblock] (Enc2) [above=15px of Enc1] {Encoder $2$};
\node (dots1) [above=15px of Enc2] {$\dots$};
\node[Eblock] (EncN) [above=15px of dots1] {Encoder $N_e$};

%Encoder Input
\node (Einp) [below=20px of Tok] {};
\node[font=\footnotesize] (Einp_dots) [right=0px of Einp] {$\dots$};
\node[Vect] (Einp_2) [left=5px of Einp_dots] {$x_2$};
\node[Vect] (Einp_1) [left=5px of Einp_2] {$x_1$};
\node[Vect] (Einp_N) [right=5px of Einp_dots] {$x_N$};
\node (txt_Einp) [left=5px of Einp_1] {Input};

%Encoder Output
\node (Eout) [above=20px of EncN] {};
\node[font=\footnotesize] (Eout_dots) [right=0px of Eout] {$\dots$};
\node[Vect] (Eout_2) [left= 5px of Eout_dots] {$z_2$};
\node[Vect] (Eout_1) [left=5px of Eout_2] {$z_1$};
\node[Vect] (Eout_N) [right=5px of Eout_dots] {$z_N$};
\node (txt_Eout) [left=5px of Eout_1] {Encoder output};

%Decoder Blocks
\node[Dblock] (Dec1) [right=75px of Enc1] {Decoder $1$};
\node[Dblock] (Dec2) [above=15px of Dec1] {Decoder $2$};
\node (dots2) [above=15px of Dec2] {$\dots$};
\node[Dblock] (DecN) [above=15px of dots2] {Decoder $N_d$};
\node[Block, align=center] (Probs) [above=15px of DecN] {Predict the probabilities\\ of the next token};

%Decoder Input
\node (Dinp) [below=20px of Dec1] {};
\node[font=\footnotesize] (Dinp_dots) [right=0px of Dinp] {$\dots$};
\node[Vect] (Dinp_2) [left=5px of Dinp_dots] {$y_2$};
\node[Vect] (Dinp_1) [left=5px of Dinp_2] {$y_1$};
\node[Vect] (Dinp_N) [right=5px of Dinp_dots] {$y_N$};
\node (txt_Dinp) [right=5px of Dinp_N] {Decoder Masked Input};

%Arrows
\draw[Fleche1] (Einp.north) to (Tok.south);
\draw[Fleche1] (Tok.north) to (EPE.south);
\draw[Fleche1] (EPE.north) to (Enc1.south);
\draw[Fleche1] (Enc1.north) to (Enc2.south);
\draw[Fleche1] (Enc2.north) to (dots1.south);
\draw[Fleche1] (dots1.north) to (EncN.south);
\draw[Fleche1] (EncN.north) to (Eout.south);

\draw[Fleche1] (Dinp.north) to (Dec1.south);
\draw[Fleche1] (Dec1.north) to (Dec2.south);
\draw[Fleche1] (Dec2.north) to (dots2.south);
\draw[Fleche1] (dots2.north) to (DecN.south);
\draw[Fleche1] (DecN.north) to (Probs.south);

\draw[Fleche2] (Eout_N.east) .. controls +(right:25px) and +(left:25px) .. (Dec1.west);
\draw[Fleche2] (Eout_N.east) .. controls +(right:25px) and +(left:25px) .. (Dec2.west);
\draw[Fleche2] (Eout_N.east) .. controls +(right:25px) and +(left:25px) .. (DecN.west);
\end{tikzpicture}
\end{center}


\newpage
\section{Structure de la partie encoder}

Soit $N$ la taille de la séquence d'entrée du modèle.

\begin{center}
\begin{tikzpicture}[
EncBlock/.style = {rectangle, rounded corners, draw=green!30, fill=green!5, thick, dashed, minimum width=150px, minimum height=212px},
Block/.style =  {rectangle,  rounded corners, draw=black!100, fill =white!0, thick, minimum width = 80px},
Ablock/.style = {rectangle, rounded corners, draw=black!40!green, fill =green!10, thick, minimum height = 50px, minimum width = 100px},
FFblock/.style = {rectangle, rounded corners, draw=black!40!blue, fill =blue!10, thick, minimum height = 50px, minimum width = 100px},
Vect/.style = {circle, draw=purple!80!red, thick, font=\footnotesize, text width=8px, align=left},
Circ/.style = {circle, draw=black, thick},
Fleche1/.style = {->, thick, shorten <=5px, shorten >= 5px},
]

%Encoder Input
\node (Einp) {};
\node[font=\footnotesize] (Einp_dots) [right=0px of Einp] {$\dots$};
\node[Vect] (Einp_2) [left=5px of Einp_dots] {$x_2$};
\node[Vect] (Einp_1) [left=5px of Einp_2] {$x_1$};
\node[Vect] (Einp_N) [right=5px of Einp_dots] {$x_N$};
\node (txt_Einp) [left=5px of Einp_1] {Input};

%Global Encoder Block
\node[EncBlock] (Enc) [above=10px of Einp] {};
\node (Txt_Encoder) [left=8px of Enc] {Encoder Block};

% MultiHead Self-Attention Block
\node[Ablock] (Att) [above=15px of Einp] {MultiHead Self-Attention};
\node[Block] (AN1) [above=15px of Att] {Add \& Normalize};

%Attention output
\node (Aout) [above=15px of AN1] {};
\node[font=\footnotesize] (Aout_dots) [right=0px of Aout] {$\dots$};
\node[Vect] (Aout_2) [left=5px of Aout_dots] {$x_2$};
\node[Vect] (Aout_1) [left=5px of Aout_2] {$x_1$};
\node[Vect] (Aout_N) [right=5px of Aout_dots] {$x_N$};

%FF block
\node[FFblock] (FF) [above=15px of Aout] {Feed Forward Layer};
\node[Block] (AN2) [above=15px of FF] {Add \& Normalize};

%Encoder Output
\node (Eout) [above=15px of AN2] {};
\node[font=\footnotesize] (Eout_dots) [right=0px of Eout] {$\dots$};
\node[Vect] (Eout_2) [left=5px of Eout_dots] {$x_2$};
\node[Vect] (Eout_1) [left=5px of Eout_2] {$x_1$};
\node[Vect] (Eout_N) [right=5px of Eout_dots] {$x_N$};
\node (txt_Eout) [left=5px of Eout_1] {Output};

%Arrows
\draw[Fleche1] (Einp.north) to (Att.south);
\draw[Fleche1] (Att.north) to (AN1.south);
\draw[Fleche1] (AN1.north) to (Aout.south);
\draw[Fleche1] (Aout.north) to (FF.south);
\draw[Fleche1] (FF.north) to (AN2.south);
\draw[Fleche1] (AN2.north) to (Eout.south);


\draw[Fleche1] (Einp_N.east) .. controls +(right:45px) and +(right:45px) ..  node[Circ,scale=0.7, fill=white] {$+$}  (AN1.east);
\draw[Fleche1] (Aout_N.east) .. controls +(right:45px) and +(right:45px) .. node[Circ,scale=0.7, fill=white] {$+$} (AN2.east);
\end{tikzpicture}
\end{center}


\newpage
\section{Schéma du Feed Forward Network :}

Soit $d_E$ la dimension des embedding des tokens. Soit $N$ la taille de la séquence d'entrée du modèle. \\

Ici, $X$ est une matrice de dimension $(N, d_E)$ donnée en entrée du réseau, à laquelle on applique une première couche linéaire : $X \mapsto X \cdot W_1^T + B_1$ pour obtenir une matrice $H$ de dimension $(D, d_E)$, avec $D$ la dimension de l'état caché du Feed Forward Network. On applique ensuite une seconde couche linéaire $H \mapsto H \cdot W_2^T + B_2$ pour obtenir la matrice de sortie $Y$ de dimension $(N, d_E)$.\\

Les matrices $W_1, W_2, B_1$ et $B_2$ sont internes au réseau et sont apprises lors de l'entraînement.

\begin{center}
\begin{tikzpicture}[
Lin/.style = {rectangle, rounded corners, draw=red!30!yellow!50, fill =yellow!10, dashed, thick, minimum height = 15px, minimum width = 150px},
Vect/.style = {circle, draw=purple!150!red, thick, font=\footnotesize, text width=8px, align=left},
Vect2/.style = {ellipse, draw=purple!150!red, thick, font=\footnotesize, text width=8px, align=left},
Circ/.style = {circle, draw=black, thick},
Trait/.style = {-, shorten <=1px, shorten >= 1px},
]


%FF Input
\node (FFinp) {};
\node[font=\footnotesize] (FFinp_dots) [right=0px of FFinp] {$\dots$};
\node[Vect] (FFinp_2) [left=10px of FFinp_dots] {$x_2$};
\node[Vect] (FFinp_1) [left=10px of FFinp_2] {$x_1$};
\node[Vect] (FFinp_N) [right=10px of FFinp_dots] {$x_N$};
\node (txt_FFinp) [left=10px of FFinp_1] {Input $X$};

%Linear1
\node[Lin] (Lin1) [above=15px of FFinp] {};
\node (Txt_lin1) [right=5px of Lin1] {Linear Layer 1};

%Hidden Layer
\node (Hl) [above= 15px of Lin1] {};
\node[font=\footnotesize] (Hl_dots) [above= 45px of FFinp] {$\dots$};
\node[Vect] (Hl_2) [left=10px of Hl_dots] {$h_2$};
\node[Vect] (Hl_1) [left=10px of Hl_2] {$h_1$};
\node[Vect2, minimum width=28px] (Hl_D1) [right=10px of Hl_dots] {$\!\!\!\! h_{D-1}$};
\node[Vect] (Hl_D) [right=10px of Hl_D1] {$h_D$};
\node (txt_Hl) [left=10px of Hl_1] {Hidden State $H$};

%Linear2
\node[Lin] (Lin2) [above=15px of Hl_dots] {};
\node (Txt_lin2) [right=5px of Lin2] {Linear Layer 2};

%FF Output
\node (FFout)  [above=15px of Lin2] {};
\node[font=\footnotesize] (FFout_dots) [right=0px of FFout] {$\dots$};
\node[Vect] (FFout_2) [left=10px of FFout_dots] {$y_2$};
\node[Vect] (FFout_1) [left=10px of FFout_2] {$y_1$};
\node[Vect] (FFout_N) [right=10px of FFout_dots] {$y_N$};
\node (txt_FFout) [left=10px of FFout_1] {Output $Y$};

%Nodes Connections
\draw[Trait] (FFinp_1.north) to (Hl_D.south);
\draw[Trait] (FFinp_1.north) to (Hl_D1.south);
\draw[Trait] (FFinp_1.north) to (Hl_2.south);
\draw[Trait] (FFinp_1.north) to (Hl_1.south);

\draw[Trait] (FFinp_2.north) to (Hl_D.south);
\draw[Trait] (FFinp_2.north) to (Hl_D1.south);
\draw[Trait] (FFinp_2.north) to (Hl_2.south);
\draw[Trait] (FFinp_2.north) to (Hl_1.south);

\draw[Trait] (FFinp_N.north) to (Hl_D.south);
\draw[Trait] (FFinp_N.north) to (Hl_D1.south);
\draw[Trait] (FFinp_N.north) to (Hl_2.south);
\draw[Trait] (FFinp_N.north) to (Hl_1.south);

\draw[Trait] (FFout_1.south) to (Hl_D.north);
\draw[Trait] (FFout_1.south) to (Hl_D1.north);
\draw[Trait] (FFout_1.south) to (Hl_2.north);
\draw[Trait] (FFout_1.south) to (Hl_1.north);

\draw[Trait] (FFout_2.south) to (Hl_D.north);
\draw[Trait] (FFout_2.south) to (Hl_D1.north);
\draw[Trait] (FFout_2.south) to (Hl_2.north);
\draw[Trait] (FFout_2.south) to (Hl_1.north);

\draw[Trait] (FFout_N.south) to (Hl_D.north);
\draw[Trait] (FFout_N.south) to (Hl_D1.north);
\draw[Trait] (FFout_N.south) to (Hl_2.north);
\draw[Trait] (FFout_N.south) to (Hl_1.north);
\end{tikzpicture}
\end{center}


\newpage
\section{Structure du réseau utilisé}

Je vais donc utiliser la partie encodeur du modèle BERT, qui va donc me donner une représentation vectorielle de la séquence donnée en entrée, que je vais pouvoir donner à un petit modèle (Deep) Feed Forward Classifier, dont la dernière couche linéaire aura pour sortie une dimension 1, dont je vais faire passer la sortie dans une fonction softmax afin de me donner une valeur $c$ entre 0 et 1 que l'on pourra interpréter de la façon suivante : plus $c$ est proche de $0$, plus le texte en entrée porte un sentiment négatif, et inversement, plus $c$ est proche de $1$, plus le texte en entrée porte un sentiment positif.

\begin{center}
\begin{tikzpicture}[
EncBlock/.style = {rectangle, rounded corners, draw=green!30, fill=green!5, thick, dashed, minimum width=150px, minimum height=212px},
Block/.style =  {rectangle,  rounded corners, draw=black!100, fill =white!0, thick, minimum width = 80px, align=center},
Bblock/.style = {rectangle, rounded corners, draw=black!40!green, fill =green!10, thick, minimum height = 50px, minimum width = 100px, align=center},
FFblock/.style = {rectangle, rounded corners, draw=black!40!blue, fill =blue!10, thick, minimum height = 50px, minimum width = 100px, align=center},
Vect/.style = {circle, draw=purple!80!red, thick, font=\footnotesize, text width=8px, align=left},
Circ/.style = {circle, draw=black, thick},
Fleche1/.style = {->, thick, shorten <=5px, shorten >= 5px},
]

%Blocks
\node (Inp) {``Input''};
\node[Block] (BT) [right=25px of Inp] {BERT\\Tokenizer};
\node[Bblock] (BE) [right=25px of BT] {BERT\\Encoder};
\node[FFblock] (FFC) [right=25px of BE] {Feed Forward\\Classifier};
\node[Vect] (Out) [right=25px of FFC] {$c$};

%Arrows
\draw[Fleche1] (Inp.east) to (BT.west);
\draw[Fleche1] (BT.east) to (BE.west);
\draw[Fleche1] (BE.east) to (FFC.west);
\draw[Fleche1] (FFC.east) to (Out.west);

\end{tikzpicture}
\end{center}






\end{document}